% !TeX spellcheck = es_ES
\renewcommand\chapterillustration{wall/wall5}	% image for each chapter
\chapter{Para qué lógica}
\begin{inicio}{}{}%
	\section*{Contenidos del capítulo}	
	\startcontents[chapters]
	\printcontents[chapters]{}{1}{\setcounter{tocdepth}{1}}
	\bigskip \bigskip
	\section*{Objetivos de aprendizaje}	
	\begin{enumerate}
	\item Que comprendas para qué sirve la lógica: por qué es importante y con qué fines se usa la argumentación, en diferentes áreas.
	\item Que conozcas la relación de la lógica con el pensamiento crítico y con otras formas de convencimiento.
	\item Que conozcas los diferentes usos del lenguaje, y puedas reconocerlos en el discurso hablado y escrito.
	\end{enumerate}
\end{inicio}\clearpage


\noindent \lettrine[lines=3]{E}{ste libro es} una introducción a distintos aspectos de la lógica. La lógica, como se dice de otras tantas disciplinas, es una ciencia y un arte. Es una \textit{ciencia} porque es un estudio sistemático y riguroso de un fenómeno concreto: la argumentación correcta (después analizaremos esta noción informal de "correcto"; de hecho, buena parte del libro va sobre eso). Involucra la aplicación de las matemáticas al estudio de este fenómeno, y se relaciona contras ciencias, como la computación teórica y la lingüística. La lógica también es un \textit{arte}, en el sentido de ser una familia de \textit{técnicas} que podemos aprender y utilizar en la vida cotidiana, así como en distintas áreas del conocimiento. Estas técnicas están enfocadas a la creación, identificación y análisis de argumentos correctos. En este libro vamos a revisar varios aspectos de estas dos caras de la lógica. Mientras tanto, sistematicemos lo dicho hasta ahora en nuestra primera definición.


\begin{definition}{Lógica}{Logica}
	La lógica es la \textit{ciencia} y el \textit{arte} de la argumentación correcta. 
\end{definition}




\bigskip
\begin{resumen}
	\section*{Resumen del capítulo}	
	\begin{list}{$^\bigstar$}{}
		\item La lógica tiene tres aspectos:
		\begin{itemize}
			\item \textit{Filosófico}: es el resultado de diversos proyectos filosóficos, entre los que se incluye la sistematización del razonamiento deductivo y la fundamentación de las matemáticas.
			\item \textit{Matemático}: es una teoría matemática abstracta que se aplica para modelar un fenómeno concreto: la argumentación.
			\item \textit{Técnico}: es una habilidad, una técnica o arte, que puede aplicarse en diferentes contextos, y que requiere de ciertos principios prudenciales para que sepamos cómo y cuándo aplicarla.
		\end{itemize}
		\item Un sistema lógico tiene varios componentes:
		\begin{itemize}
			\item Lenguaje,
			\item Teoría de modelos,
			\item Teoría de la demostración.
		\end{itemize}
		\item Además de los sistemas lógicos que revisaremos en este libro, existen sistemas de \textit{lógica no clásica}. Estos se suelen dividir en:
		\begin{itemize}
			\item Extensiones,
			\item Subsistemas,
			\item Rivales.
		\end{itemize}
	\end{list}
\end{resumen} \bigskip

%\newpage
%\begin{tcolorbox}[enhanced,breakable,colframe=white,interior style={top color=blue!10,bottom color=white},before upper={\parindent15pt}]
%	\section*{Proyecto del capítulo: }
%	\paragraph*{Antecedentes $\dashrightarrow$} Antecedentes
%	\paragraph*{Problema $\dashrightarrow$} Problema
%\end{tcolorbox}
\printendnotes

%\stopcontents[chapters]


% \chapter*{Resumen de la Parte I}

% \noindent La filosofía es una disciplina argumentativa, por lo que parte esencial de su método es el análisis de la argumentación. Un argumento es una secuencia de proposiciones, de las cuales la última es la conclusión y las primeras son las premisas; ambas se relacionan mediante la \textit{inferencia}. Las inferencias pueden ser deductivas o probabilísticas; en este libro estudiaremos a las deductivas. La lógica que estudiaremos es un \textit{modelo matemático} de uno de los aspectos de la argumentación: la inferencia deductiva. En particular, el objetivo principal de la lógica formal es entender la validez de los argumentos: qué significa que un argumento sea (o no sea) válido, qué características lo hacen válido, y cómo demostrar que un argumento es válido. Para ello, primero aprenderemos la Lógica Clásica de Orden Cero, y después la Lógica Clásica de Primer Orden. Estos son sistemas lógicos, que tienen un lenguaje, una semántica formal y un aparato deductivo. Con ellos, modelaremos argumentos del lenguaje natural, para poder aplicar técnicas matemáticas al entendimiento de la argumentación en filosofía y en la vida cotidiana.